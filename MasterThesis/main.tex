%Plantilla basada en "Template for Masters / Doctoral Thesis" (plantilla disponible en writeLaTex) que subió LaTeXTemplates.com

\documentclass[11pt]{book}
\usepackage[paperwidth=17cm, paperheight=22.5cm, bottom=2.5cm, right=2.5cm]{geometry}
\usepackage{amssymb,amsmath,amsthm} %paquete para símbolo matemáticos
\usepackage[spanish]{babel}
\usepackage[utf8]{inputenc} %Paquete para escribir acentos y otros símbolos directamente
\usepackage{enumerate}
\usepackage{graphicx}
%\usepackage{subfig} %para poner subfiguras
\graphicspath{{Img/}} %En qué carpeta están las imágenes
\usepackage[nottoc]{tocbibind}
\usepackage[pdftex,
            pdfauthor={NOMBRE DEL AUTOR},
            pdftitle={TÍTULO DE LA TESIS},
            pdfsubject={ÁREA DE LA TESIS},
            pdfkeywords={PALABRAS CLAVE},
            pdfproducer={Latex con hyperref},
            pdfcreator={pdflatex}]{hyperref}



\begin{document}

%----------------------------------------------------------------------------------------
%	COMANDOS PERSONALIZADOS
%----------------------------------------------------------------------------------------

%SI TU TESIS TIENE TEOREMAS Y DEMOSTRACIONES, PUEDES DESCOMENTAR Y USAR LOS SIGUIENTES COMANDOS

%\renewcommand{\proofname}{Demostración}
%\providecommand{\norm}[1]{\lVert#1\rVert} %Provee el comando para producir una norma.
%\providecommand{\innp}[1]{\langle#1\rangle} 
%\newcommand{\seno}{\mathrm{sen}}
%\newcommand{\diff}{\mathrm{d}}

%\newtheorem{teo}{Teorema}[section] 
%\newtheorem{cor}[teo]{Corolario}
%\newtheorem{lem}[teo]{Lema}

%\theoremstyle{definition}
%\newtheorem{dfn}[teo]{Definición}

%\theoremstyle{remark}
%\newtheorem{obs}[teo]{Observación}

%\allowdisplaybreaks


%----------------------------------------------------------------------------------------
%	PORTADA
%----------------------------------------------------------------------------------------

\title{TÍTULO DE LA TESIS} %Con este nombre se guardará el proyecto en writeLaTex

\begin{titlepage}
\begin{center}

\textsc{\Large Instituto Tecnológico Autónomo de México}\\[4em]

%Figura
\begin{figure}[h]
\begin{center}
\includegraphics{logo-ITAM_ch.jpg}
\end{center}
\end{figure}

\vspace{4em}

\textsc{\huge \textbf{TÍTULO DE LA TESIS}}\\[4em]

\textsc{\large Tesis}\\[1em]

\textsc{que para obtener el título de}\\[1em]

\textsc{TÍTULO  VAS A OBTENER}\\[1em]

\textsc{presenta}\\[1em]

\textsc{\Large AUTOR}\\[1em]

\textsc{\large Asesor: NOMBRE}

\end{center}

\vspace*{\fill}
\textsc{México, D.F. \hspace*{\fill} 2014}

\end{titlepage}


%----------------------------------------------------------------------------------------
%	DECLARACIÓN
%----------------------------------------------------------------------------------------

\thispagestyle{empty}
\vspace*{\fill}
\begingroup
``Con fundamento en los artículos 21 y 27 de la Ley Federal del Derecho de Autor y como titular de los derechos moral y patrimonial de la obra titulada ``\textbf{TÍTULO DE LA TESIS}'', otorgo de manera gratuita y permanente al Instituto Tecnológico Autónomo de México y a la Biblioteca Raúl Bailléres Jr., la autorización para que fijen la obra en cualquier medio, incluido el electrónico, y la divulguen entre sus usuarios, profesores, estudiantes o terceras personas, sin que pueda percibir por tal divulgación una contraprestación''.

\centering

\hspace{3em}

\textsc{AUTOR}

\vspace{5em}

\rule[1em]{20em}{0.5pt} % Línea para la fecha

\textsc{Fecha}
 
\vspace{8em}

\rule[1em]{20em}{0.5pt} % Línea para la firma

\textsc{Firma}

\endgroup
\vspace*{\fill}


%----------------------------------------------------------------------------------------
%	DEDICATORIA
%----------------------------------------------------------------------------------------

\pagestyle{empty}
\frontmatter

\chapter*{}
\begin{flushright}
\textit{DEDICATORIA}
\end{flushright}


%----------------------------------------------------------------------------------------
%	AGRADECIMIENTOS
%----------------------------------------------------------------------------------------

\chapter*{Agradecimientos}
%\markboth{AGRADECIMIENTOS23}{AGRADECIMIENTOS} % encabezado 

¡Muchas gracias a todos!


%----------------------------------------------------------------------------------------
%	PREFACIO
%----------------------------------------------------------------------------------------

\chapter*{Prefacio}

\pagestyle{plain}
%\markboth{PREFACIO23}{PREFACIO} % encabezado 

PUEDEN QUITAR ESTA PARTE


%----------------------------------------------------------------------------------------
%	TABLA DE CONTENIDOS
%---------------------------------------------------------------------------------------

\tableofcontents


%----------------------------------------------------------------------------------------
%	TESIS
%----------------------------------------------------------------------------------------
\mainmatter %empieza la numeración de las páginas
\pagestyle{headings}

%  Incluye los capítulos en el folder de capítulos

\include{Capitulos/cap1}
\thispagestyle{empty}
\include{Capitulos/Conclusiones}
\thispagestyle{empty}


%----------------------------------------------------------------------------------------
%	APÉNDICES
%----------------------------------------------------------------------------------------

\addtocontents{toc}{\vspace{2em}} % Agrega espacios en la toc

\appendix % Los siguientes capítulos son apéndices

%  Incluye los apéndices en el folder de apéndices

\include{Apendices/Ap}
\thispagestyle{empty}
%\include{Apendices/AppendixB}
%\include{Apendices/AppendixC}

\addtocontents{toc}{\vspace{2em}} % Agrega espacio en la toc


%----------------------------------------------------------------------------------------
%	BIBLIOGRAFÍA
%----------------------------------------------------------------------------------------
\backmatter
\nocite{*}
\bibliographystyle{plain}
\bibliography{bibliografía.bib} %Aquí ponen el nombre del archivo .bib






\end{document}