%----------------------------------------------------------------------------------------
%	RESUMEN (1 hoja)
%----------------------------------------------------------------------------------------

\pagestyle{empty}
\chapter {Resumen}

El vídeo y la fotografía es, hoy en día, una parte imprescindible del entretenimiento en  realidad virtual, pero tiene grandes desventajas frente a otras tecnologías, como las experiencias en entornos tridimensionales, debido a la falta de interactividad.

Desde hace unos años empresas como Disney, Facebook y Google han establecido un precedente en la industria tecnológica, buscando establecer los llamados seis grados de libertad al vídeo y la fotografía 360º para mejorar la sensación de inmersión. Existen numerosas maneras de conseguir este objetivo y cada una tiene sus ventajas y sus inconvenientes.

El propósito del trabajo descrito en esta memoria es implementar, de la manera más realista y eficiente posible, un sistema que proporcione seis grados de libertad en un rango de movimiento limitado. Para ello se va a recurrir a un algoritmo basado en desplazamientos pixel a pixel, teniendo capacidad de manipulación de la información al nivel mas detallado disponible, haciendo uso de todos los recursos disponibles en los dispositivos incluyendo hardware gráfico con propósito general.

La forma de conseguir este objetivo será evaluar muchas implementaciones para, más tarde, seleccionar los mejores resultados y trabajar sobre ellos con la finalidad de hacerlos más eficientes y realistas. Todo esto sin dejar de contemplar más opciones como las que han implementado otros desarrolladores.