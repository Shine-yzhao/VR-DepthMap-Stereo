%----------------------------------------------------------------------------------------
%	PLANTEAMIENTO DEL PROBLEMA (1 hoja)
%----------------------------------------------------------------------------------------

\pagestyle{empty}
\chapter {Planteamiento del problema}

La fotografía y el vídeo para realidad virtual actualmente tienen muchas limitaciones, tanto para producirlo como para visualizarlo. Una de estas limitaciones y del cual este proyecto se ocupa es la falta de libertad movimiento. 

La foto y el vídeo solo tienen tres grados de libertad y de ellos cabeceo o \textit{roll} (inclinar la cabeza sobre los hombros) no funciona como cabría esperar y puede provocar desde mareos hasta ver las imágenes duplicadas. La captura de imágenes, tanto reales como virtuales, se hace con cámaras que irremediablemente están en un punto concreto del espacio y eso en principio limita el movimiento del usuario.

En mayo de 2017, Google dio una charla aportando que cerca del 50\% del tiempo pasado en Daydream se centra en experiencias de vídeo. Un año después Facebook en el F8 \cite{FBOculusGo}, hablando sobre el estudio para el diseño de Oculus Go, puso de manifiesto que el 99\% de los usuarios consumen vídeo y que el 83\% de tiempo utilizado se destina a multimedia, llegando a la conclusión que es uno de los casos de uso principales.

Por ello, este trabajo se centra en mejorar la visualización del vídeo con un sistema que permita al usuario desplazarse físicamente dentro de un área, reaccionando el vídeo a ese posicionamiento en tiempo real.