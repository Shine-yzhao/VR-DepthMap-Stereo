%----------------------------------------------------------------------------------------
%	PLANTEAMIENTO DEL PROBLEMA (1 hoja)
%----------------------------------------------------------------------------------------

\pagestyle{empty}
\chapter {Planteamiento del problema}

La fotografía y el vídeo para realidad virtual actualmente tienen muchas limitaciones tanto para producirlo como para visualizarlo. Esto hace que el problema se pueda intentar solucionar desde los dos extremos siendo estas soluciones en muchos casos compatibles y complementarias entre ellas.

Los principales problemas consisten en una baja sensación de profundidad provocado por una mala captura de imagen y en parte por la falta de libertad movimiento. Se podría considerar que existen tan solo 2 grados de libertad reales que son el viraje y la inclinación ya que el cabeceo no funciona como cabría esperar y puede provocar desde mareos hasta ver las imágenes duplicadas. Además como las cámaras están fijas, el movimiento no está permitido.

Todo esto aumenta la fricción en la experiencia de usuario que provoca que se reduzca el consumo de este tipo de material o incluso decida no volver en un corto plazo lo que tiene como consecuencia que las empresas no inviertan en este tipo de productos.

Desde el punto de vista de la producción existen soluciones en las que se están trabajando, desde dispositivos especializados a métodos de postproducción que mejoran los problemas de ensamblado de las imágenes capturadas. Algunos de estos métodos tienen un coste asequible, pero los que mejores resultados dan como los llamados campos de luz, requieren una inversión demasiado alta para la mayoría de producciones de realidad virtual.

Por ello, este trabajo se centra en mejorar la visualización del vídeo proponiendo soluciones software que puedan utilizarse en tiempo real.