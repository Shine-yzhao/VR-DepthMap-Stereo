%----------------------------------------------------------------------------------------
%	INTRODUCCIÓN (2,3 hojas)
%----------------------------------------------------------------------------------------

\pagestyle{empty}
\chapter{Introducción}
% tema de trabajo
La realidad virtual es una tecnología que desde hace unos años ha estado intentando hacerse un hueco en la industria del entretenimiento.

Actualmente la realidad virtual tiene una variedad muy grande de dispositivos  con diferentes características. Las principales especificaciones a tener en cuenta a la hora de decidir utilizar uno en concreto son la disponibilidad de accesorios como los mandos, la resolución de pantalla y los grados de libertad tanto de las gafas como de los periféricos, siendo típicos 3 y 6 grados.

Los grados de libertad definen la capacidad de movimiento que tiene un elemento. En el caso de 3 grados de libertad en realidad virtual, hace referencia a los giros sobre el eje principal (viraje o \textit{yaw}, inclinación o \textit{pitch} y cabeceo o \textit{roll}), mientras que cuando se amplia a 6 grados de libertad hace referencia al desplazamiento en los tres ejes.

La realidad virtual está en auge pero que sin embargo todavía esta construyendo una identidad propia. El contenido que se genera todavía esta basado en gran parte en técnicas ya conocidas como reproducción de vídeo y fotos cuya máxima adaptación consiste simplemente en poner una imagen ligeramente diferente en cada ojo.

Este trabajo trata de conseguir proporcionar a los usuarios de experiencias de Realidad Virtual mayor inmersión a la hora de ver contenidos que no están siendo generados en vivo, sino que han sido creados previamente ya sea con una cámara real o gráficos por ordenador.

La característica principal de este tipo de contenido es que cada imagen esta tomada desde un punto fijo en el espacio. Esto provoca una problemática que consiste en que el usuario únicamente tiene 2 grados de libertad reales a la hora de visualizarlo en unas gafas de realidad virtual que son el viraje y la inclinación, ya que cuando el cabeceo es grande provoca ver imágenes duplicadas y puede provocar incomodidad o incluso mareo cuando existe un pequeño cabeceo.

% interés
El principal motivo para realizar esta investigación es reducir la fricción que tiene el usuario con este tipo de contenido y así facilitar la retención y fidelización.

Durante el desarrollo del proyecto se llevan a cabo pruebas con diferentes técnicas y se evalúa la viabilidad en diferentes dispositivos, el realismo del resultado así como la escalabilidad de los métodos y el ámbito de acción de los mismos.















