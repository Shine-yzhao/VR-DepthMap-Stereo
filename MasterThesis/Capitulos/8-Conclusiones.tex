%----------------------------------------------------------------------------------------
%	CONCLUSIONES (2 hojas)
%----------------------------------------------------------------------------------------

\pagestyle{empty}
\chapter {Conclusiones}

El vídeo y la fotografía es hoy en día una parte imprescindible del entretenimiento en  realidad virtual. 

Sin embargo, tanto el vídeo como la fotografía son modelos de entretenimiento muy pobres en comparación con las experiencias interactivas que se ofrecen utilizando modelos tridimensionales. Por todo esto es importante preocuparse por mejorar estos medios y trabajar por ofrecer mejores soluciones.

Durante este proyecto se han estado trabajando con soluciones que hacen uso de algoritmos con una gran carga de trabajo, con su consecuente pérdida de rendimiento en los dispositivos de menores especificaciones. Para poder encontrar el algoritmo que mejor funciona hay que probar muchas implementaciones e ir descartando las soluciones menos prácticas.

Durante el desarrollo de las diferentes aproximaciones, encontré una serie de problemas a los que tuve que dar solución. Por ejemplo, utilizar un shader gráfico no era una opción viable para el programa.

Por otro lado, el código tiene partes donde se encontraron problemas que habitualmente se resuelven depurándolos, como es el caso de la función de desplazamiento en espacio de mundo. La depuración de shaders que se ejecutan sobre hardware gráfico es muy complicada de llevar a cabo y generalmente debe ser resuelta sacando colores por pantalla, cosa que no siempre es todo lo intuitivo que cabría esperar.

El objetivo propuesto tiene mucho fondo, pues el último propósito no estaba planteado para conseguirlo en el plazo de un trabajo de fin de máster, sino que todavía queda mucho trabajo por hacer en este campo. Aún así creo que se ha conseguido realizar una aproximación que permita continuar el desarrollo con nuevas características como el rellenado de huecos mediante imágenes que se encuentren paralelas a la actual y la sustitución de la vista actual por estas imágenes si el desplazamiento es superior a cierta cantidad.

Este trabajo a su vez ha requerido un gran componente de investigación, pues hay muchas y grandes empresas como Disney, Facebook y Google trabajando para hacer posible esta idea. Al ser empresas punteras en tecnología, lo hace un campo de trabajo muy complicado.

Como resultado de este trabajo puedo decir que las tecnologías de video e imagen en realidad virtual tienen mucho camino por recorrer pero que están encaminadas para que los creadores de contenido tengan una vía más completa de realizar sus obras.

Todavía queda mucho por hacer pero creo que no está muy lejos el día que, mientras tenemos unas gafas de realidad virtual puestas, podamos movernos alrededor de los personajes de nuestra película favorita y podamos sentirnos parte de ella.



